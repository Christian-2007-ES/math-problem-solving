\documentclass[12pt,a4paper]{article}

% Paquetes útiles
\usepackage[utf8]{inputenc}
\usepackage[T1]{fontenc}
\usepackage{amsmath, amssymb, amsfonts}
\usepackage{geometry}
\usepackage{hyperref}
\usepackage{enumitem}
\geometry{margin=2.5cm}

% Título
\title{Problem Solving – Matemáticas}
\author{Christian Torres}
\date{\today}

\begin{document}
	
	\maketitle
	
	\section{Problema 1}
	Sean $a,b,c$ números reales positivos. Probar la desigualdad
	\begin{center}
		$\dfrac{1}{a(b+1)}+\dfrac{1}{b(c+1)}+\dfrac{1}{c(a+1)}\geq\dfrac{3}{1+abc}.$
	\end{center}
	
	\textbf{Solución:} Usando la notación de sumatorio cíclico la desigualdad se convierte en
	\begin{center}
		$\displaystyle\sum_{cyc}\dfrac{1}{a(b+1)}\geq\dfrac{3}{1+abc}$.
	\end{center}
	 Ahora por ser $a,b,c>0$, tenemos que
	\begin{center}
		$\displaystyle\sum_{cyc}\dfrac{1}{a(b+1)}\geq\dfrac{3}{1+abc}\Leftrightarrow (1+abc)\displaystyle\sum_{cyc}\dfrac{1}{a(b+1)}=\sum_{cyc}\dfrac{1+abc}{a(b+1)}\geq3\Leftrightarrow\sum_{cyc}\dfrac{1+abc}{a(b+1)}+1\geq6\Leftrightarrow\sum_{cyc}\dfrac{1+abc+ab+a}{a(b+1)}=\sum_{cyc}\dfrac{1+a+ab(1+c)}{a(b+1)}\geq6\Leftrightarrow\sum_{cyc}\dfrac{1+a}{a(b+1)}+\dfrac{b(1+c)}{b+1}\geq6$.
	\end{center}
	Basta probar la ultima desigualdad, por la desigualdad entre las media aritmética y geométrica tenemos que
	\begin{center}
		$\displaystyle\sum_{cyc}\dfrac{1+a}{a(b+1)}+\dfrac{b(1+c)}{b+1}\geq6\sqrt[6]{\prod_{cyc}\dfrac{(1+a)b(1+c)}{a(b+1)^2}}=6\sqrt[6]{\dfrac{abc(a+1)^2(b+1)^2(c+1)^2}{abc(a+1)^2(b+1)^2(c+1)^2}}=6$.
	\end{center}
\end{document}